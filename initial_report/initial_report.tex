% PLEASE FILL IN THE PLACEHOLDERS <...>
%
% Diplomarbeit/Studienarbeit/IDP von <NAME>
% Diploma thesis of <NAME>
%
% Title: <TITLE>
%        <TITLE>
%
\documentclass[12pt,a4paper]{article}

%%%%%%%%%%%%%%%%%%%%%%%%%%%%%%%%%%%%%%%%%%%%%%%%%%%%%%%%%%%%

% PACKAGES:
% Define typearea
% a) Use automatic:
\usepackage[BCOR1cm]{typearea}

% Use German :
\usepackage[german, english]{babel}
% Use list of tabels, etc. in table of contents:
\usepackage{tocbibind}
% German paragraph skip
\usepackage{parskip}
% Encoder:????
%\usepackage[utf-8]{inputenc}
\usepackage[utf8]{inputenc}
% Use A4-paper efficiently:
\usepackage{a4wide}
% Index-generation
\usepackage{makeidx}
% Einbinden von URLs:
\usepackage{url}
% Include .eps-files (needed also for the LKN-logo):
%\usepackage{epsf}
\usepackage{epsfig}
\usepackage{epstopdf}
% Special \LaTex symbols (e.g. \BibTeX):
\usepackage{doc}
% Include Graphic-files:
%\usepackage{graphics}
% Include Graphic-files:
\usepackage{graphicx}
\usepackage{pgfgantt}
% Include doc++ generated tex-files:
%\usepackage{docxx}
% Include PDF links
\usepackage[pdftex, bookmarks=true]{hyperref}

\definecolor{barblue}{RGB}{153,204,254}
\definecolor{groupblue}{RGB}{51,102,254}
\definecolor{linkred}{RGB}{165,0,33} 
%%%%%%%%%%%%%%%%%%%%%%%%%%%%%%%%%%%%%%%%%%%%%%%%%%%%%%%%%%%%

% OTHER SETTINGS:

% Pagestyle:
\pagestyle{headings}

% Avoid 'overhang':
\sloppy

% Choose language
\newcommand{\setlang}[1]{\selectlanguage{#1}\nonfrenchspacing}

%%%%%%%%%%%%%%%%%%%%%%%%%%%%%%%%%%%%%%%%%%%%%%%%%%%%%%%%%%%%

% TITLE:

\begin{document}


\thispagestyle{empty}
\newpage

\vspace{5cm}

\parbox{15cm}{\begin{center} {\sf\bf 
                               \Large  Technische Universität München
                                \smallskip

                               \Large Lehrstuhl für Datenverarbeitung
                               \smallskip
                              }

                            
              \end{center}}  %&

\vspace{5cm}

\begin{center}
        {\bf\Huge Applied Reinforcement Learning Project} % Studienarbeit, Interdisziplinäres Projekt
\end{center}

\begin{center}
        \settowidth{\baselineskip}{0.4cm}
        {\LARGE 
		-Title-
	    \vspace*{3.5cm}
        %\LARGE Alperen Gündogan 
        }
\end{center}

\begin{center}
\LARGE Alperen Gündogan, Rachid Ellouze, Uzair Akbar
\end{center}

\begin{center}
\large\today
\end{center}

\vspace*{3.2cm}
\begin{center}
    \epsfxsize=3cm
    \epsfbox{LDVLogoCMYK_oT.pdf}
    \hspace*{9cm}
    \epsfxsize=3cm
    \epsfbox{TUMLogo_oZ_Vollfl_CMYK.pdf}
\end{center}


%%%%%%%%%%%%%%%%%%%%%%%%%%%%%%%%%%%%%%%%%%%%%%%%%%%%%%%%%%%%

% MAIN PART:

%\include{Abstract}

\setlang{english}
\thispagestyle{plain}

\section{Motivation}


\section{Goals}


\section{Project Steps}


\section{Time Plan}

\begin{figure}
%\begin{center}
%\begin{preview}

\noindent\resizebox{\textwidth}{!}{
%\begin{tikzpicture}[x=.5cm, y=1cm]
    \begin{ganttchart}[
	vgrid={*{6}{draw=none},dotted},  % vgrid, 
    hgrid = true,
    bar height=.2,
    x unit = .07cm,
    y unit title=.6cm,
    y unit chart=.4cm,
    milestone/.append style={fill=green},
    bar/.append style={fill=red},
    time slot format=isodate,
    ]{2018-11-01}{2019-04-30}  % <---
      \gantttitlecalendar{year, month=name}\\
       \ganttgroup{WP-1}{2018-11-01}{2018-11-25} \\
       \ganttbar{T1.1}{2018-11-01}{2018-11-15} \\
       \ganttbar{T1.2}{2018-11-15}{2018-11-25} \\
       \ganttgroup{WP-2}{2018-11-25}{2018-12-20}\\
       \ganttbar{T2.1}{2018-11-25}{2018-12-10} \\
       \ganttbar{T2.2}{2018-12-10}{2018-12-20} \\
       \ganttmilestone{MS-1}{2018-12-20}\\
       \ganttgroup{WP-3}{2018-12-20}{2019-01-15}\\
       \ganttbar{T3.1}{2018-12-20}{2019-01-5} \\ 
       \ganttbar{T3.2}{2019-01-5}{2019-01-15} \\
       \ganttgroup{WP-4}{2019-01-015}{2019-04-15}  \\
       \ganttbar{T4.1}{2019-01-15}{2019-03-15}  \\
       \ganttmilestone{MS-2}{2019-03-15} \\
      \ganttbar{T4.2}{2019-03-15}{2019-04-15} \\
      \ganttgroup{WP-5}{2018-11-25}{2019-04-30} \\
      \ganttbar{T5.1}{2018-11-25}{2019-04-01} \\
      \ganttbar{T5.2}{2019-04-01}{2019-04-30} \\
      \ganttmilestone{MS-3}{2019-04-30}    
    \end{ganttchart}
}
%\end{center}
%\caption{Gantt Chart}
\end{figure}
%\end{preview}
%
%\begin{itemize}
%\item Work Package 1: Thorough literature research. Identification and understanding the relevant work.
%
%	\begin{enumerate}
%	\item Study on 5G - V2X architecture(release 14, TS 23.285).
%	\item Listing of classical resource management algorithms for LTE and 5G.
%	\end{enumerate}
%	
%\item Work Package 2: Identify, test and learn about the required software libraries.
%	\begin{enumerate}
%	\item Learning the protocol stack of 5G - V2X and testing possible V2X examples with the software.
%	\item Developing a strategy(e.g. flowchart) to implement resource management algorithms into the simulation(\textit{Milestone 1}).
%	\end{enumerate}
%	
%\item Work Package 3: Testing of mostly used 2-3 algorithms in the simulation environment and evaluation of their strengths and weaknesses.
%	\begin{enumerate}
%	\item Implementation of the determined algorithms into the simulation.
%	\item Creating plots: probability of V2V links that satisfy latency requirements vs number of V2V links. Throughput vs number of V2V links. Impact of various allocation time intervals(e.g. 20ms, 100ms etc.).
%	\end{enumerate}
%	
%\item Work Package 4: Design, specification and implementation of a DRL algorithm.
%	\begin{enumerate}
%	\item Design and implementation of an algorithm, testing for different scenarios.(\textit{Milestone 2)}.
%	\item Utilization of the algorithm for better results.
%	\end{enumerate}
%
%\item Work Package 5: Writing final report.
%	\begin{enumerate}
%	\item Writing a literature review, algorithm specifications.
%	\item Comparison of all the results and evaluation of the DRL algorithm(\textit{Milestone 3}).
%	\end{enumerate}
%
%\end{itemize}
%
%

%\vspace*{9.2cm}

\subsection{Contengency plan}
%Cellular V2X is a very novel topic, therefore software libraries and tools are still at the developing stage. One possible problem might be lackness or underdevelopment of some required libraries for the latest C-V2X architecture. In this situation, one should develop short part of the library to be able to implement and test resource management algorithms. 



% References (Literaturverzeichnis):
% a) Style (with numbers: use unsrt):
\bibliographystyle{ieeetr}
% b) The File:
\bibliography{Bibliography}




%%%%%%%%%%%%%%%%%%%%%%%%%%%%%%%%%%%%%%%%%%%%%%%%%%%%%%%%%%%%
\end{document}
